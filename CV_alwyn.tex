%%%%%%%%%%%%%%%%%%%%%%%%%%%%%%%%%%%%%%%%%
% Long Professional Curriculum Vitae
% LaTeX Template
% Version 1.1 (9/12/12)
%
% This template has been downloaded from:
% http://www.latextemplates.com
%
% Original author:
% Rensselaer Polytechnic Institute (http://www.rpi.edu/dept/arc/training/latex/resumes/)
%
% Important note:
% This template requires the res.cls file to be in the same directory as the
% .tex file. The res.cls file provides the resume style used for structuring the
% document.
%
%%%%%%%%%%%%%%%%%%%%%%%%%%%%%%%%%%%%%%%%%

\documentclass[12pt]{res} % Use the res.cls style, the font size can be changed to 11pt or 12pt here


\usepackage{helvet} % Default font is the helvetica postscript font
%\usepackage{newcent} % To change the default font to the new century schoolbook postscript font uncomment this line and comment the one above

\newsectionwidth{0pt} % Stops section indenting

\usepackage{hyperref} % url

\begin{document}

\name{\Large Alwyn Mathew}

\address{
	\centerline{Kochumalil (H), Wallardie Jn, Vandiperiyar PO, Idukki Dist, Kerala-685533, India} \\  
	\centerline {Email: alwyn.pcs16@iitp.ac.in, alwynmathew.90@gmail.com} \\
	\centerline {LinkedIn: \url{www.linkedin.com/in/alwynmathew}} \\
	\centerline {Personal page: \url{https://alwynm.github.io}} \\
	\centerline{Mobile: +91 96452 39304} \\} 

\begin{resume}

\hrule
\vspace{0.2in} 

Ph.D. fellow trained in Computer Vision with areas of expertise in 3D Computer Vision. Passionate about teaching machines to see like we do. I have extensive research experience and the ability to work independently or as part of a team.

\vspace{0.1in}
\section{\centerline{EDUCATION}} 
\vspace{0.1in} % Gap between title and text

\textbf {Indian Institute of Technology (IIT) Patna} \hfill  Patna, India \\ 
{\sl Ph.D. in Computer Vision} \hfill Aug 2021 

\textbf {College of Engineering Karunagappally} \hfill  Kerala, India \\ 
{\sl Master’s in Technology, Image Processing} \hfill May 2016

\textbf {Tamilnadu College Of Engineering} \hfill  Coimbatore, India \\ 
{\sl Bachelor's in Technology, Information Technology} \hfill April 2012

\vspace{0.1in}
\section{\centerline{SCHOLARSHIPS AWARDED}} 
\vspace{0.1in} 

Spsorship from on \textbf{Scheme for Promotion of Academic and Research Collaboration}, Ministry of Human Resource development, Government of India. Grant \#P582. \hfill April 2020

Three year \textbf{Senior Research Fellowship} (SRF) at IIT Patna \\
Awarded by Ministry of Human Resource Development, Government of India. \hfill April 2018

Two year \textbf{Junior Research Fellowship} (JRF)  at IIT Patna \\
Awarded by Ministry of Human Resource Development, Government of India. \hfill July 2016

Two year \textbf{Post Graduate Fellowship}  at College of Engineering Karunagappally \\
Institute of Human Resources Development, Government of Kerala.\hfill July 2014

\vspace{0.1in}
\section{\centerline{COMPETITIVE AWARDS}} 
\vspace{0.1in} 

\textbf{Second Place}, IoT Grand Challenge, Indian Institute of Technology (IIT) Patna. \hfill 2016

\textbf{Finalist}, Bosch DNA Challenge, Bosch India. \hfill 2017

\textbf{Top 35}, Patna Ideathon, Government of Bihar. \hfill 2018

\section{\centerline{PROFESSIONAL MEMBERSHIP}} 
\vspace{0.1in} 

\textbf{IEEE Student Member}, IEEE Membership Number: 96850267.

\vspace{0.02in}
\section{\centerline{COMPETITIVE EXAMINATIONS QUALIFIED}} 
\vspace{0.1in} 
Department of Higher Education, Ministry of Human Resources Development, India

\begin{itemize}
\setlength\itemsep{-1.0em}
\item Graduate Aptitude Test in Engineering (\textbf{GATE}) 2016,  All India Rank: 5493 out of 108495 candidates (95 percentile).
\end{itemize}

\section{\centerline{RESEARCH EXPERIENCE}} 

\vspace{0.1in} 

\textbf {Doctoral  Research Experience} \hfill  Patna, India \\ 
{\sl Indian Institute of Technology Patna} \hfill July 2016--May 2021 
\vspace{5pt}
\begin{itemize}

\item Developed expertise in Camera Models.
\item Developed expertise in self-supervised depth estimation from a single camera.
\item Introduced direct depth estimation with a distorted camera lens.
\item Studied the impact of self-attention in depth estimation network. 

\item Developed expertise in Adversarial Samples and their effect on deep neural networks.
\item Studied the vulnerabilities of monocular depth estimators against Adversarial attacks.

\item Introduced an intelligent agent for shifting load from no-peak to off-peak hours in residential grids.
\item Studied the complexity of the RL-DSM environment and improved the learning curve of the agent.

\item Presented results at departmental seminars to more than 30 attendees. 
\end{itemize}

\textbf {Ongoing Research works} \hfill Patna, India \\ 
{\sl Collaboration with my Thesis supervisor {\rm (Dr. Jimson Mathew)}} \hfill Since May 2021 
\vspace{5pt}
\begin{itemize}
\item Dynamic moving object masking for monocular depth estimation with video sequence.
\item Handle textureless surface in photometric loss
\item Light-weight monocular depth network
\end{itemize}


\textbf {Experience in Research Guidance} \hfill Patna, India \\ 
{\sl Indian Institute of Technology Patna} \hfill Since July 2017

Mentored Junior Research Fellows in the group.
\begin{itemize}
\item Fisheye cameras are commonly used in applications like autonomous driving and surveillance to provide a large field of view ($>180^{\circ}$). We developed per-pixel dense distance estimation on fisheye cameras for automotive scenes. 
\item Deep learning-based load prediction model on time series data. These models will be used for applications like Demand Side Management in Smart Grid.
\item Designed algorithm to adapt classification task on unlabelled data with fewer know labelled data. 
\end{itemize}


Mentored M.Tech. students in Computer Science Department.
\begin{itemize}
\item Designed a system that distinguishes familiar/unfamiliar images from EEG (Electroencephalogram) captured using an eight electrode helmet. Extended future to deception detection using deep learning.
\item We designed a fast multi-object hybrid tracking system using particle filter and neural network.
\item We designed a light-weight deep learning-based facial recognition system.
\end{itemize}

Mentored B.Tech. students in Engineering Physics. 
\begin{itemize}
\item An advanced reinforcement learning-based system for load shifting in a residential grid.
\item We developed a reinforcement learning-based system for load shifting in a residential grid. 
\item We developed deep learning-based light-weight object detection for embedded systems.
\item We have developed a system that estimates depth from a single uncalibrated camera.
\end{itemize}

\textbf {Master's Project Research Experience} \hfill  Kerala, India \\ 
{\sl College of Engineering Karunagappally} \hfill May 2015--May 2016 
\vspace{5pt}
\begin{itemize}
\item Investigated super-resolution with Convolutional Neural Networks.
\item Super-resolution in gray scale and color.
\item Presented the final report to five member evaluation committee.
\end{itemize}



\section{\centerline{TEACHING ASSISTANCE EXPERIENCE}}

\vspace{5pt}

\textbf {Indian Institute of Technology Patna} \hfill  Patna, India \\ 
{\sl Research Fellow} \hfill July 2016--July 2021 

I have performed the following teaching assistance during my Ph.D. program at Indian Institute of Technology Patna.

\vspace{6pt}
\begin{itemize}\setlength\itemsep{0em}
	
\item CS 225 Switching Theory \hfill Jan-May, 2017
\item CS 229 Innovation Laboratory \hfill Jan, 2017
\item CS 421 Computer Peripherals and Interfacing \hfill July-Dec, 2017
\item CS 225 Switching Theory \hfill Jan-May, 2018
\item CS 421 Computer Peripherals and Interfacing \hfill July-Dec, 2018
\item CS 225 Switching Theory \hfill Jan-May, 2019
\item EE 541 High Performance Computing \hfill Jan, 2019
\item CS 421 Computer Peripherals and Interfacing \hfill July-Dec, 2019
\item CS 225 Switching Theory \hfill Jan-May, 2020
\item CS 421 Computer Peripherals and Interfacing \hfill July-Dec, 2020
\item Mid and End-Semester Examination duties \hfill 2016--2020

\end{itemize}

\section{\centerline{SKILLS}}

\vspace{10pt}
\begin{table}[ht]
	\begin{tabular}{ll}
		\textbf{Coding} & Python, C++, Java, ASP.NET, C\# .NET. \\
		\textbf{Teaching} & Conducted B.Tech and M.Tech classes at IIT Patna \\
		\textbf{ML Packages} & Pytorch, TensorFlow, Keras \\
	\end{tabular}
\end{table}

\section{\centerline{PUBLICATIONS}}

\vspace{0.3in}

\centerline{\textbf{Journal Articles}}

\vspace{0.2in}

\begin{itemize}
\setlength\itemsep{5pt}

\item \textbf{Mathew, A.} and Mathew J., Monocular depth estimation with SPN loss, \textit{Elsevier Image and Vision Computing}, (2020).

\item \textbf{Mathew, A.}, Roy, A.,  and Mathew, J., Intelligent Residential Energy Management System Using Deep Reinforcement Learning, \textit{IEEE Systems Journal}, (2020).

\item \textbf{Mathew, A.}, Jolly, MJ., and Mathew, J., Improved Residential Energy Management System Using Priority Double Deep Q-learning, \textit{Elsevier Sustainable Cities and Society}, (2021).

\item \textbf{Mathew, A.}, and Mathew, J., MDDNet: Learn Depth and Ego-motion from Videos with Camera Distortion, \textit{Elsevier Computer Vision and Image Understanding}, (2020). \textit{(Under-revision)}

\item \textbf{Mathew, A.}, Patra, A., and Mathew, J., Monocular Depth Estimators: Vulnerabilities and Attacks, \textit{IEEE Intelligent Systems}, (2020). \textit{(Under-review)}

\item \textbf{Mathew, A.}, and Mathew, J., Monocular Depth Estimation with Unknown Camera, \textit{Elsevier Image and Vision Computing}, (2021). \textit{(Under-review)}

\item \textbf{Mathew, A.}, and Mathew, J., Efficient Demand Response in Residential Grid using Q-learning, \textit{IEEE Systems Journal}, (2021). \textit{(Under-review)}

\item \textbf{Mathew, A.}, Gopugari, B., and Mathew, J., Monocular Depth Estimation with Stereo Assistance Depth Consistency, \textit{IEEE Intelligent Systems}, (2021). \textit{(Under-review)}

\end{itemize}

\vspace{0.3in}

\centerline{\textbf{Conference Proceedings}}

\vspace{0.2in}

\begin{itemize}
	\setlength\itemsep{5pt}
	
	\item \textbf{Mathew, A.}, Patra, AP., and Mathew, J., Self-Attention Dense Depth Estimation Network for Unrectified Video Sequences, \textit{IEEE International Conference on Image Processing}, (2020).
			
	\item Sanodiya RK, \textbf{Mathew, A.}, Mathew, J., and Khushi, M., Statistical and Geometrical Alignment using Metric Learning in Domain Adaptation, \textit{IEEE International Joint Conference on Neural Networks}, (2020).
	
	\item Srivastava, H., \textbf{Mathew, A.}, and Mathew, J., A Novel Frame Similarity Based Pedestrian Counting Approach in Surveillance Videos, \textit{IEEE India Council International Conference}, (2018).
	
	\item \textbf{Mathew, A.}, Mathew, J., Govind, M., and Mooppan, A., An Improved Transfer learning Approach for Intrusion Detection, \textit{International Conference on Advances in Computing  Communication}, (2017).
	
\end{itemize}

\section{\centerline{PATENT}}

\vspace{0.3in}

\begin{itemize}
	\setlength\itemsep{5pt}
	
	\item Easa Z., Gupta D., Mathew J., and \textbf{Mathew A.}, Automated two wheeler parking system by detecting the location of the vehicle using sensor under the platform Appl.no. 201731036379. (Indian Patent Pending) \hfill 2017
	
\end{itemize}

\section{\centerline{TALKS}}

\vspace{0.3in}

\begin{itemize}
	\setlength\itemsep{5pt}
	
	\item \textbf{Generative Adversarial Networks and Adversarial Attacks} sponsored by All India Council for Technical Education (AICTE), Government of India. \hfill Dec, 2020
	
	\item \textbf{Adversarial Machine Learning} sponsored by APJ Abdul Kalam Technological \\ University, Government of Kerala. \hfill Dec, 2019
	
	\item \textbf{Machine Learning makes Smart Grids smarter} sponsored by Scheme for Promotion of Academic and Research Collaboration (SPARC), Ministry of Human Resource development, Government of India. \hfill Sept, 2019
	
	\item \textbf{Generative Adversarial Networks} sponsored by Third phase of Technical Education Quality Improvement Programme, Government of India and Institute of Electrical and Electronics Engineers. \hfill July, 2019
	
	\item \textbf{Generative Adversarial Networks and Adversarial examples} sponsored by Third phase of Technical Education Quality Improvement Programme (TEQIP-III), Government of India. \hfill July, 2019
	
	\item \textbf{Introduction to Convolutional Neural Networks} sponsored by Third phase of Technical Education Quality Improvement Programme (TEQIP-III), Government of India. \hfill Dec, 2018
	
\end{itemize}

\section{\centerline{PROFESSIONAL ACTIVITIES}}

\vspace{0.3in}

\begin{itemize}
	\setlength\itemsep{5pt}
	
	\item \textbf{Subreviewer} of IEEE International Conference on Smart Computingand Communications (ICSCC). \hfill 2017
	
	\item \textbf{Subreviewer} of IEEE International Symposium on Electronic System Design. \hfill 2018
	
	\item \textbf{Reviewer} of IET Computer Vision. \hfill 2019
	
	\item \textbf{Reviewer} of IEEE International Conference on Data Science and Engineering. \hfill 2019
	
\end{itemize}

\section{\centerline{OTHER ACTIVITIES}}

\vspace{0.3in}

\begin{itemize}
	\setlength\itemsep{5pt}
	
	\item \textbf{Technical committee} of Research Scholars’ Day, Indian Institute of Technology (IIT) Patna \hfill 2017-2019

\end{itemize}

\section{\centerline{REFERENCE}} 

\vspace{0.2in}
 
\begin{itemize}

\item \textbf {Dr. Jimson Mathew}\\ Head, Associate Professor\\ Department of Computer Science and Engineering, \\ Indian Institute of Technology Patna \\ Bihar, India\\
Email: jimson@iitp.ac.in \\
Phone: +91-612-3028347 \\
Mobile: +91-91109-56262

\item \textbf {Dr. Samrat Mondal}\\ Assistant Professor\\ Department of Computer Science and Engineering, \\ Indian Institute of Technology Patna \\ Bihar, India\\
Email: samrat@iitp.ac.in \\
Phone: +91-612-3028163 \\
Mobile: +91-82925-83635

\vspace{6pt}
\item \textbf {Dr. Binu VP}\\ Associate Professor\\ Department of Computer Science and Engineering \\ College of Engineering Karunagappally \\ Kerala, India \\
Email: binuvp@gmail.com \\
Mobile: +91-98473-90760

\end{itemize}


\end{resume} 
\end{document}