%%%%%%%%%%%%%%%%%%%%%%%%%%%%%%%%%%%%%%%%%%%%%%%%%%%%%%%%%%%%%%%%%%%%%%%%%%%%%%%%
% Medium Length Graduate Curriculum Vitae
% LaTeX Template
% Version 1.2 (3/28/15)
%
% This template has been downloaded from:
% http://www.LaTeXTemplates.com
%
% Original author:
% Rensselaer Polytechnic Institute 
% (http://www.rpi.edu/dept/arc/training/latex/resumes/)
%
% Modified by:
% Daniel L Marks <xleafr@gmail.com> 3/28/2015
% 
% Further modified by:
% Alwyn Mathew <alwynmathew.90@gmail.com> 9/20/2016
%
% Important note:
% This template requires the simple_style.cls file to be in the same directory 
% as the .tex file. The res.cls file provides the resume style used for 
% structuring the document.
%
%%%%%%%%%%%%%%%%%%%%%%%%%%%%%%%%%%%%%%%%%%%%%%%%%%%%%%%%%%%%%%%%%%%%%%%%%%%%%%%%

%-------------------------------------------------------------------------------
%	PACKAGES AND OTHER DOCUMENT CONFIGURATIONS
%-------------------------------------------------------------------------------

%%%%%%%%%%%%%%%%%%%%%%%%%%%%%%%%%%%%%%%%%%%%%%%%%%%%%%%%%%%%%%%%%%%%%%%%%%%%%%%%
% You can have multiple style options the legal options ones are:
%
%   centered:	the name and address are centered at the top of the page 
%				(default)
%
%   line:		the name is the left with a horizontal line then the address to
%				the right
%
%   overlapped:	the section titles overlap the body text (default)
%
%   margin:		the section titles are to the left of the body text
%		
%   11pt:		use 11 point fonts instead of 10 point fonts
%
%   12pt:		use 12 point fonts instead of 10 point fonts
%
%%%%%%%%%%%%%%%%%%%%%%%%%%%%%%%%%%%%%%%%%%%%%%%%%%%%%%%%%%%%%%%%%%%%%%%%%%%%%%%%

\documentclass[mm]{simple_style}  

% Default font is the helvetica postscript font
\usepackage{helvet}
\usepackage{hyperref}
\usepackage{url}
\usepackage{xcolor}
\hypersetup {
    colorlinks=true,
    linkcolor=colorlink,
    filecolor=magenta,      
    urlcolor=colorlink,
}
\usepackage[left=0.7in, right=2in, top=0.9in]{geometry}

% Increase text height
\textheight=700pt

\begin{document}

%-------------------------------------------------------------------------------
%	NAME AND ADDRESS SECTION
%-------------------------------------------------------------------------------
\name{Alwyn Mathew}
\qualification{Postdoctoral Research Assistant, University of Dundee}
\emailone{alwynmathew.90@gmail.com}
%\emailone{alwyn.pcs16@iitp.ac.in} \emailtwo{alwynmathew.90@gmail.com}
\website{https://alwynm.github.io/}{alwynm.github.io}
\linkedin{https://www.linkedin.com/in/alwynmathew}{linkedin.com/in/alwynmathew}
\github{https://github.com/alwynmathew}{github.com/alwynmathew}
\phone{+91--96452--39304} 

\address{Division of Imaging Science and Technology\\ School of Medicine, University of Dundee, UK}
%\address{Kochumalil~(H), Wallardie Jn, Vandiperiyar PO\\ Idukki Dist, Kerala--685533, India}
%\address{B3-R512, IIT Patna, Bihta, Patna, Bihar--801103, India}
%-------------------------------------------------------------------------------

\begin{resume}
	
%-------------------------------------------------------------------------------
%	EXPERIENCE SECTION
%-------------------------------------------------------------------------------
\section{Work\\Experience}
\cusemph{University of Dundee}, Scotland, United Kingdom \timeline{Dec 2021--present}\\
{\sl Postdoctoral Research Assistant} \\
\sl{Division of Imaging Science and Technology}

\sectionline
%-------------------------------------------------------------------------------

%-------------------------------------------------------------------------------
%	EDUCATION SECTION
%-------------------------------------------------------------------------------
\section{Education}
\cusemph{Indian Institute of Technology (IIT) Patna}, Bihar, India \timeline{Aug 2021}\\
{\sl Ph.D. in Computer Vision} \\
\cusemph{CGPA: 8.63/10}

\cusemph{College of Engineering Karunagappally}, Kerala, India \timeline{May 2016}\\
{\sl Master’s in Technology, Image Processing} \\
\cusemph{CGPA: 8.8/10}

\cusemph{Tamilnadu College Of Engineering}, Coimbatore, India \timeline{April 2012}\\
{\sl Bachelor’s in Technology, Information Technology} 

\sectionline
%-------------------------------------------------------------------------------

%-------------------------------------------------------------------------------
%	RESEARCH SECTION
%-------------------------------------------------------------------------------
\section{Research\\Interests}
\par
Computer Vision, 3D Computer Vision \\
Medical Robotics, Autonomous Robots \\
Adversarial Machine Learning, Adversarial Attacks\\
Reinforcement Learning, Demand Side Management\\
%-------------------------------------------------------------------------------
%      PUBLICATIONS 
%-------------------------------------------------------------------------------
\halfsectionline
\vspace{-8mm}
\section{\href{https://alwynm.github.io/pub}{Publications}}

\vspace{-3.5ex}
\subsection{Journal Articles}
\vspace{-2ex}

\cusemph{Mathew, A.} and Mathew J., Monocular depth estimation with SPN loss, \textit{Image and Vision Computing}, 2020.

\cusemph{Mathew, A.}, Roy, A.,  and Mathew, J., Intelligent Residential Energy Management System Using Deep Reinforcement Learning, \textit{IEEE Systems Journal}, 2020.

\cusemph{Mathew, A.}, Jolly, MJ., and Mathew, J., Improved Residential Energy Management System Using Priority Double Deep Q-learning, \textit{Sustainable Cities and Society}, 2021.

\cusemph{Mathew, A.}, and Mathew, J., MDDNet: Learn Depth and Ego-motion from Videos with Camera Distortion, \textit{Computer Vision and Image Understanding}, 2020. \textit{(Under-revision)}

\cusemph{Mathew, A.}, Patra, A., and Mathew, J., Monocular Depth Estimators: Vulnerabilities and Attacks, \textit{IEEE Intelligent Systems}, 2020. \textit{(Under-review)}

\cusemph{Mathew, A.}, and Mathew, J., Monocular Depth Estimation with Unknown Cameras, \textit{Knowledge Based System}, 2021. \textit{(Under-review)}

\cusemph{Mathew, A.}, and Mathew, J., Cheaper Depth Sensor Enhances
Monocular Depth Estimation, \textit{Information Fusion}, 2021. \textit{(Under-review)}

\cusemph{Mathew, A.}, and Mathew, J., Efficient Demand Response in Residential Grid using Q-learning, \textit{IEEE Systems Journal}, 2021. \textit{(Under-review)}

\cusemph{Mathew, A.}, Gopugari, B., and Mathew, J., Improving Unsupervised Monocular Depth
Estimation with Stereo Assistance, \textit{IEEE Transactions on Intelligent Vehicles}, 2021. \textit{(Under-review)}
\halfsectionline
\vspace{-12mm}
\subsection{Conference Proceedings}
\vspace{-2ex}

Bander, B., \cusemph{Mathew, A.}, Magerand, L., Trucco, E., and Manfredi, L., Real-Time Lumen Detection for Autonomous Colonoscopy, \textit{MICCAI 2022 ISGIE Workshop}, 2022.

\cusemph{Mathew, A.}, Neeraj., and Mathew, J., UnkDisp: Monocular Depth Estimation with
Unknown Baseline and Focal Length, \textit{International Conference on Neural Information Processing}, 2022. \textit{(Under-review)}

\cusemph{Mathew, A.}, Patra, AP., and Mathew, J., Self-Attention Dense Depth Estimation Network for Unrectified Video Sequences, \textit{IEEE International Conference on Image Processing}, 2020.

Sanodiya RK, \cusemph{Mathew, A.}, Mathew, J., and Khushi, M., Statistical and Geometrical Alignment using Metric Learning in Domain Adaptation, \textit{IEEE International Joint Conference on Neural Networks}, 2020.

Srivastava, H., \cusemph{Mathew, A.}, and Mathew, J., A Novel Frame Similarity Based Pedestrian Counting Approach in Surveillance Videos, \textit{IEEE India Council International Conference}, 2018.

\cusemph{Mathew, A.}, Mathew, J., Govind, M., and Mooppan, A., An Improved Transfer learning Approach for Intrusion Detection, \textit{International Conference on Advances in Computing  Communication}, 2017.\\
\halfsectionline
\vspace{-12mm}
\subsection{Patents}
\vspace{-2ex}

Easa Z., Gupta D., Mathew J., and \cusemph{Mathew A.}, Automated two wheeler parking system by detecting the location of the vehicle using sensor under the platform Appl.no. 201731036379. (2017 Indian Patent Pending)

\vspace{-2ex}
\sectionline

%-------------------------------------------------------------------------------

\section{Research\\Experience}

\begin{subproject}	
	\title{Post-Doctoral  Research Experience \hfill  Scotland, UK} 
	\supervisor{Supervisor: Dr. Luigi Manfredi (University of Dundee)}
	\duration{Since Dec 2021}
	\description{
		- Developed real-time \textbf{lumen detection} model for autonomous colonoscopy. \\
		- Developed \textbf{supervised depth estimation} models for endoscopy monocular camera. \\
		- Developed self-supervised depth estimation models for \textbf{high FOV} endoscopy monocular camera. \\
		- Developed expertise in \textbf{integrating} deep learning models with soft robot. \\
		- Developed expertise in \textbf{real time} deep learning models.
	} 
\end{subproject}
\halfsectionline

\begin{subproject}	
  \title{Doctoral  Research Experience \hfill  Patna, India} 
  \supervisor{Supervisor: Dr. Jimson Mathew (IIT Patna)}
  \duration{July 2016 -- May 2021}
  \description{
  	- Developed expertise in \textbf{camera models}. \\
  	- Developed expertise in self-supervised \textbf{depth estimation} from a single camera.\\
  	- Introduced direct depth estimation with a \textbf{distorted} camera lens.\\
  	- Studied the impact of \textbf{self-attention} in depth estimation network. \\
  	- Developed expertise in \textbf{adversarial samples} and their effect on deep neural networks.\\
  	- Studied the \textbf{vulnerabilities} of monocular depth estimators against Adversarial attacks.\\
  	- Introduced an intelligent agent for \textbf{shifting load} from no-peak to off-peak hours in residential grids.\\
  	- Studied the complexity of the RL-DSM environment and improved the learning curve of the agent.\\
  	- Presented results at departmental seminars to more than 30 attendees. \\
  } 
\end{subproject}
%\halfsectionline
%\begin{subproject}
%  \title{Ongoing Research \hfill  Patna, India}
%  \supervisor{Collaboration with my Thesis supervisor (Dr. Jimson Mathew)}
%  \duration{Since May 2021}
%  \description{
%  	- \textbf{Dynamic moving object} masking for monocular depth estimation with video sequence.\\
%  	- Handle \textbf{textureless} surface in photometric loss.\\
%  	- \textbf{Light-weight} monocular depth network.\\
%  }
%\end{subproject}
\halfsectionline
\begin{subproject}
	\title{
		\href{https://alwynm.github.io/teaching\#mentorship}{Experience in Research Guidance} \hfill  Patna, India}
	\supervisor{Indian Institute of Technology Patna}
	\duration{Since July 2017}
	\description{
		\cusemph{Mentored Junior Research Fellows}\\
		- Fisheye cameras are commonly used in applications like autonomous driving and surveillance to provide a large field of view. We developed per-pixel dense distance estimation on fisheye cameras for automotive scenes. \\
		- Deep learning-based load prediction model on time series data. These models will be used for applications like Demand Side Management in Smart Grid.\\
		- Designed algorithm to adapt classification task on unlabelled data with fewer know labelled data. \\
		
		\cusemph{Mentored M.Tech. students in Computer Science Department}\\
		- Designed a system that distinguishes familiar/unfamiliar images from EEG (Electroencephalogram) captured using an eight electrode helmet. Extended future to deception detection using deep learning.\\
		- We designed a fast multi-object hybrid tracking system using particle filter and neural network.\\
		- We designed a light-weight deep learning-based facial recognition system.\\
	}
%\end{subproject}
%\begin{subproject}
%  \title{}
%  \supervisor{}
%  \duration{}
  \description{			
	\cusemph{Mentored B.Tech. students in Computer Science Department}\\
		- An advanced reinforcement learning-based system for load shifting in a residential grid.\\
		- We developed a reinforcement learning-based system for load shifting in a residential grid. \\
		- We developed deep learning-based light-weight object detection for embedded systems.\\
		- We have developed a system that estimates depth from a single uncalibrated camera.
	}
\end{subproject} %\vspace{-4ex}
%\begin{subproject}
%	\title{}
%	\supervisor{}
%	\duration{}
%	\description{			
%		\cusemph{Mentored interns}\\
%		- We build a  stereo data collection rig with multiple IP cameras and one Kinect sensor.\\
%		- We build a fast data collection pipeline from Microsoft Kinect 360. \\
%		- We developed vision based fast crowd density monitoring system.\\
%	}
%\end{subproject}
%\halfsectionline
\begin{subproject}
  \title{Master's Project Research Experience \hfill  Kerala, India}
  \supervisor{Supervisor: Dr. Binu VP}
  \duration{May 2015--May 2016}
  \description{
	- Investigated super-resolution with Convolutional Neural Networks.\\
	- Super-resolution in gray scale and color.\\
	- Presented the final report to five member evaluation committee.\\
  }
\end{subproject}

\vspace{-2ex}
\sectionline
%-------------------------------------------------------------------------------

\section{\href{https://alwynm.github.io/teaching\#teaching-assistant}{Teaching Experience}}

\begin{project}	
	\title{Teaching Assistance \hfill  IIT Patna, India} 
	\supervisor{}
	\duration{July 2016--Dec 2020}
	\description{
		CS 225 Switching Theory \hfill Jan--May, 2017 \\
		CS 229 Innovation Laboratory \hfill Jan, 2017 \\
		CS 421 Computer Peripherals and Interfacing \hfill July--Dec, 2017 \\
		CS 225 Switching Theory \hfill Jan--May, 2018 \\
		CS 421 Computer Peripherals and Interfacing \hfill July--Dec, 2018 \\
		CS 225 Switching Theory \hfill Jan--May, 2019 \\
		EE 541 High Performance Computing \hfill Jan, 2019 \\
		CS 421 Computer Peripherals and Interfacing \hfill July--Dec, 2019 \\
		CS 225 Switching Theory \hfill Jan--May, 2020 \\
		CS 421 Computer Peripherals and Interfacing \hfill July--Dec, 2020 \\
		Mid and End-Semester Examination duties \hfill 2016--2020
	} 
\end{project}

\vspace{-6ex}
\sectionline
%-------------------------------------------------------------------------------
%       AWARDS & ACHIEVEMENTS	
%-------------------------------------------------------------------------------
\section{Awards \& Achievements}
\vspace{-3ex}
\subsection{Scholarships \& Sponsorship}
\vspace{-2ex}

Sponsorship from \cusemph{Scheme for Promotion of Academic and Research Collaboration}, Ministry of Human Resource development, Government of India. Grant \#P582. \timeline{April 2020} 

Three year \cusemph{Senior Research Fellowship} (SRF) at IIT Patna, \\
Ministry of Human Resource Development, Government of India. \timeline{April 2018}

Two year \cusemph{Junior Research Fellowship} (JRF)  at IIT Patna, \\
Ministry of Human Resource Development, Government of India. \timeline{July 2016}

Two year \cusemph{Post Graduate Fellowship} at College of Engineering Karunagappally \\
Institute of Human Resources Development, Government of Kerala. \timeline{July 2014}
\halfsectionline
\vspace{-12mm}
\subsection{Competitive Awards}
\vspace{-2ex}

\cusemph{Second Place}, IoT Grand Challenge, Indian Institute of Technology (IIT) Patna. \timeline{2016}

\cusemph{Finalist}, Bosch DNA Challenge, Bosch India. \timeline{2017}

\cusemph{Top 35}, Patna Ideathon, Government of Bihar. \timeline{2018} %\newpage
\halfsectionline
\vspace{-12mm}
\subsection{Competitive Examinations}
\vspace{-2ex}

Graduate Aptitude Test in Engineering (\cusemph{GATE}) \timeline{2016} \\ All India Rank: 5493 out of 108495 candidates. 

\vspace{-2ex}
\sectionline
%-------------------------------------------------------------------------------
%	ACADEMIC PROJECTS SECTION
%-------------------------------------------------------------------------------
\section{\href{https://alwynm.github.io/talks}{Talks}}

\cusemph{Generative Adversarial Networks and Adversarial Attacks} \timeline{Dec, 2020} \\
sponsored by All India Council for Technical Education, Government of India. 

\textbf{Adversarial Machine Learning}  \timeline{Dec, 2019} \\
sponsored by APJ Abdul Kalam Technological University, Government of Kerala.

\cusemph{Machine Learning makes Smart Grids smarter} \timeline{Sept, 2019} \\
sponsored by Scheme for Promotion of Academic and Research Collaboration (SPARC), Ministry of Human Resource development, Government of India. 

\cusemph{Generative Adversarial Networks} \timeline{July, 2019} \\ 
sponsored by Third phase of Technical Education Quality Improvement Programme, Government of India and IEEE.

\cusemph{Generative Adversarial Networks and Adversarial examples} \timeline{July, 2019} \\
sponsored by Third phase of Technical Education Quality Improvement Programme, Government of India. 

\cusemph{Introduction to Convolutional Neural Networks} \timeline{Dec, 2018} \\
sponsored by Third phase of Technical Education Quality Improvement Programme, Government of India.

\vspace{-2ex}
\sectionline
%-------------------------------------------------------------------------------

%-------------------------------------------------------------------------------
%	COURSE PROJECTS SECTION
%-------------------------------------------------------------------------------
\section{Professional \\Activities}


\cusemph{Reviewer} of MICCAI. \timeline{2022}\\
\cusemph{IEEE Student Member}, IEEE Membership Number: 96850267. \timeline{2019--2021}\\
\cusemph{Reviewer} of IEEE International Conference on Data Science and Engineering. \timeline{2019}\\
\cusemph{Reviewer} of IET Computer Vision. \timeline{2019}\\
\cusemph{Subreviewer} of IEEE International Symposium on Electronic System Design. \timeline{2018}\\
\cusemph{Subreviewer} of IEEE ICSCC. \timeline{2017}\\ \\
\cusemph{Organizer}, MICCAI Imaging Systems for GI Endoscopy Workshop.  \timeline{Sept 2022}\\
\cusemph{Organizer}, GIAN course on Reliable and Fault Tolerent Computing. \timeline{Jan 2017} \\
\cusemph{Organizer}, GIAN course on Internet of Things Security: Issues, Innovations, \timeline{Dec 2016}\\and Interplays.

\vspace{-2ex}
\sectionline
%-------------------------------------------------------------------------------

%-------------------------------------------------------------------------------
%	COMPUTER SKILLS SECTION
%-------------------------------------------------------------------------------
\section{Skills}

\begin{table}[ht]
	\begin{tabular}{ll}
		\cusemph{Coding} & Python, C++, Java, ASP.NET, C\# .NET \\
		\cusemph{ML Packages} & Pytorch, TensorFlow, Keras \\
		\cusemph{Teaching} & Conducted B.Tech and M.Tech classes at IIT Patna \\
	\end{tabular}
\end{table}

\vspace{-2ex}
\sectionline
%-------------------------------------------------------------------------------

%-------------------------------------------------------------------------------
%	Interests
%-------------------------------------------------------------------------------
\section{References}

\textbf {Dr. Jimson Mathew}\\ 
Head \& Associate Professor (PhD supervisor)\\ 
Department of Computer Science and Engineering \\ 
Indian Institute of Technology Patna \\ 
Bihar, India\\
Email: jimson@iitp.ac.in \\
Phone: +91-612-3028347 \\
Mobile: +91-91109-56262 \\
\halfsectionline \vspace{-5mm}

\textbf {Dr. Samrat Mondal}\\ 
Assistant Professor (Doctoral Committee Chairman)\\ 
Department of Computer Science and Engineering \\ 
Indian Institute of Technology Patna \\ Bihar, India\\
Email: samrat@iitp.ac.in \\
Phone: +91-612-3028163 \\
Mobile: +91-82925-83635 \\
\halfsectionline \vspace{-5mm}

\textbf {Dr. Abyayananda Maiti}\\ 
Assistant Professor (Doctoral Committee Member)\\ 
Department of Computer Science and Engineering \\ 
Indian Institute of Technology Patna \\ Bihar, India\\
Email: abyaym@iitp.ac.in \\
Phone: +91-612-3028130 \\
Mobile: +91-70708-11668 \\
\sectionline

%-------------------------------------------------------------------------------
\end{resume}
\end{document}
